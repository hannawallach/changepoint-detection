\documentclass[10pt,english,oneside]{article}
\usepackage[pdftex]{graphics}
\usepackage{amsmath,amssymb,rotating,multirow}
\usepackage{palatino}
\usepackage{url}

\usepackage{subfigure}

\usepackage{natbib}

\usepackage[plain]{algorithm}
\usepackage{algpseudocode}

\usepackage{parskip}
\usepackage{setspace}
\onehalfspacing

\usepackage{geometry}
\geometry{verbose,letterpaper}

\setlength{\parskip}{10pt}

\title{Changepoint Detection}
\author{}
\date{\today}

\bibliographystyle{apalike}

\newcommand{\comment}[1]{}

% bold letters and symbols

\newcommand{\ba}{\boldsymbol{a}}
\newcommand{\bb}{\boldsymbol{b}}
\newcommand{\bc}{\boldsymbol{c}}
\newcommand{\bm}{\boldsymbol{m}}
\newcommand{\bn}{\boldsymbol{n}}
\newcommand{\bt}{\boldsymbol{t}}
\newcommand{\bw}{\boldsymbol{w}}
\newcommand{\bx}{\boldsymbol{x}}
\newcommand{\by}{\boldsymbol{y}}
\newcommand{\bz}{\boldsymbol{z}}

\newcommand{\bN}{\boldsymbol{N}}

\newcommand{\bzero}{\boldsymbol{0}}

\newcommand{\bchi}{\boldsymbol{\chi}}
\newcommand{\bphi}{\boldsymbol{\phi}}
\newcommand{\btheta}{\boldsymbol{\theta}}

% distributions

\newcommand{\Dir}{\text{Dir}\,}
\newcommand{\Mult}{\text{Mult}\,}

\newcommand{\g}{\,|\,}
\newcommand{\teq}{\!=\!}
\newcommand{\is}{:=}

\renewcommand{\algorithmiccomment}[1]{\% #1}

\newcommand{\argmax}{\operatornamewithlimits{argmax}}


\begin{document}

\maketitle

\section{Bayesian Online Changepoint Detection}

A changepoint is an abrupt variation in the parameters of a generative
model. This note describes the framework of~\citet{adams07bayesian},
as applied to Bayesian unigram language modeling.

We observe words at $T$ discrete points in time $1 \leq t \leq T$. We
denote the observation (word) at time $t$ as $x_t$. We denote the
observations between times $a$ and $b$ inclusive as $\bx_{a:b}$. We
assume that the observations $\bx_{1:T}$ can be divided into
non-overlapping runs of contiguous observations. We assume that the
observations within a single run $\rho$ were drawn independently and
identically from a probability distribution with parameters
$\bphi_{\rho}$. We futher assume that the parameters for run $\rho$
are independent of and identically distributed to the parameters for
any other run $\rho'$.

The ``run length'' at time $t$, denoted by $0 \leq r_t \leq t-1$, is
the number of observations in $\bx_{1:t-1}$ that belong to the same
run as observation $x_t$. We denote this set of observations, i.e.,
$x_{t-r_t}, \ldots, x_{t-1}$, as $\bx_{t-r_t: t-1}$ for any $r_t$ and
define $\bx_{t:t-1} \teq \emptyset$. If $r_t \teq 0$ then $x_t$ is the
first observation in a new run; otherwise $x_t$ is the $(r_t +
1)^{\textrm{th}}$ observation in the current run. In other words,
$r_t$ is the length of the current run \emph{after} observation
$x_{t-1}$ and \emph{before} observation $x_t$. Changepoints therefore
occur \emph{between} observations: if $r_t \teq 0$, then a changepoint
occurred between observation $x_{t-1}$ and observation $x_t$.

\begin{algorithm}[t]
  \caption[]{Generating discrete observations.}
  \label{alg:generative_process}
  \begin{algorithmic}[0]
    \State $C \is \emptyset$
    \State $\rho \is 0$
    \For{$t \is 1$ to $T$}
    \If{$t \teq 1$}
    \State $r_t \is 0$
    \Else
    \If{$U(0,1) < H(r_{t-1}+1)$}
    \State $r_t \is 0$
    \Else
    \State $r_t \is r_{t-1}+1$
    \EndIf
    \EndIf
    \If{$r_t = 0$}
    \State $C \is C \cup \{ t \}$
    \State $\rho \is \rho + 1$
    \State $\bphi_{\rho} \sim \Dir(\bphi_{\rho}; \beta \bn)$
    \EndIf
    \State $x_t \sim \bphi_{\rho}$
    \EndFor
  \end{algorithmic}
\end{algorithm}

\comment{\subsection{Calculating $P(r_t \g \bx_{1:t-1})$}

At every point in time $1 \leq t \leq T$, we are interested in the
probabilities of all possible run lengths at that point given
observations $\bx_{1:t-1}$---i.e., $P(r_t \g \bx_{1:t-1})$ for all $0
\leq r_t \leq t-1$. The probability of $r_t \teq 0$ given observations
$\bx_{1:t-1}$ is the probability of a changepoint after observation
$x_{t-1}$ and before observation $x_t$. The most probable run length
at time $t$ is $\argmax_{0 \leq r_t \leq t-1} P(r_t \g \bx_{1:t-1})$.}

\subsection{Calculating $P(r_t \g \bx_{1:t})$}

For every point in time $1 \leq t \leq T$, we are interested in the
posterior probabilities of all possible run lengths at that point
given observations $\bx_{1:t}$---i.e., $P(r_t \g \bx_{1:t})$ for all
$0 \leq r_t \leq t-1$:
\begin{align}
  P(r_t \g \bx_{1:t}) &\propto P(r_t, x_t \g \bx_{1:t-1})\\
  &= P(x_t \g r_t, \bx_{1:t-1})\,P(r_t \g \bx_{1:t-1}).
\end{align}
The posterior probability of $r_t \teq 0$ is the posterior probability
of a changepoint after observation $x_{t-1}$ and before observation
$x_t$. The most probable run length is $\argmax_{0 \leq r_t \leq t-1}
P(r_t \g \bx_{1:t})$.

If we define $P(r_1 \teq 0 \g \bx_{1:0}) = P(r_1 \teq 0 \g \emptyset)
= P(r_1 \teq 0) = 1$ then we can recursively compute the conditional
probability of run length $r_t$ given observations $\bx_{1:t-1}$ for
all $2 \leq t \leq T$ as follows:
\begin{align}
  P(r_t \g \bx_{1:t-1})
  &\propto P(r_t, x_{t-1} \g \bx_{1:t-2})\\
  &= \sum_{r_{t-1}=0}^{(t-1)-1} P(r_{t-1}, r_t, x_{t-1} \g
  \bx_{1:t-2})\\
  &=
  \sum_{r_{t-1}=0}^{(t-1)-1} \underbrace{P(r_t \g r_{t-1}, x_{t-1},
  \bx_{1:t-2})}_{{} = P(r_t \g r_{t-1})}\,P(r_{t-1}, x_{t-1} \g
  \bx_{1:t-2})\\
  \label{eqn:recursion}
  &=   \sum_{r_{t-1}=0}^{(t-1)-1} P(r_t \g r_{t-1})\,\underbrace{P(x_{t-1} \g r_{t-1},
  \bx_{1:t-2})}_{{} = P(x_{t-1} \g \bx_{t-1-r_{t-1}:t-2})}\,P(r_{t-1} \g \bx_{1:t-2})
\intertext{where}
\label{eqn:hazard}
P(r_t \g r_{t-1}) &= \begin{cases} H(r_{t-1}+1) & \textrm{if $r_t
    \teq 0$}\\
1 - H(r_{t-1}+1) & \textrm{if $r_t \teq r_{t-1} + 1$}\\
0 & \textrm{otherwise.}\end{cases}
\end{align}
If $1 \leq r_t \leq t-1$ then $r_{t-1}$ must be equal to $r_t -
1$. Equation~\ref{eqn:recursion} therefore simplifies to
\begin{equation}
\label{eqn:simplification}
  P(r_t \g \bx_{1:t-1}) \propto
  P(r_t \g r_{t-1} \teq r_t - 1)\,\underbrace{P(x_{t-1} \g
  r_{t-1} \teq r_t - 1, \bx_{1:t-2})}_{{} = P(x_{t-1} \g
    \bx_{t-r_t:t-2})}\,P(r_{t-1} \teq r_t - 1 \g
  \bx_{1:t-2}).
  \end{equation}
Combining equations~\ref{eqn:recursion},~\ref{eqn:hazard},
and~\ref{eqn:simplification} gives
\begin{align}
  &P(r_t \g \bx_{1:t-1}) \notag\\
  &\quad\propto \begin{cases}
\sum_{r_{t-1}=0}^{(t-1)-1} H(r_{t-1}+1)\,P(x_{t-1} \g
\bx_{t-1-r_{t-1}:t-2})\,P(r_{t-1} \g
\bx_{1:t-2}) & \textrm{if $r_t \teq 0$}\\
    (1-H(r_{t-1}+1 \teq r_t))\, P(x_{t-1} \g
    \bx_{t-r_t:t-2})\,P(r_{t-1} \teq r_t - 1 \g
  \bx_{1:t-2}) & \textrm{$1 \leq r_t \leq t-1$.}
\end{cases}
  \end{align}
Pseudocode for calculating $P(r_t \g \bx_{1:t-2})$ for all $1 \leq t
\leq T$ and $0 \leq r_t \leq t-1$ is given in
algorithms~\ref{alg:conditional} and~\ref{alg:conditional2}.

\begin{algorithm}[t]
  \caption[]{$R[r_t, t-1]$ is the conditional probability of run
    length $r_t$ given observations $\bx_{1:t-1}$. For example,
    $R[2,3] = P(r_4 \teq 2 \g \bx_{1:3})$ because $t - 1 \teq 3
    \implies t \teq 4$ and $r_4 \teq 2$. $P[r_{t-1}, t-2]$ is the
    predictive probability of observation $x_{t-1}$ given observations
    $\bx_{t-1-r_{t-1}:t-2}$.}
  \label{alg:conditional}
  \begin{algorithmic}[0]
    \State \Comment{Base case (i.e., initialization)...}
    \Statex
    \State $R[0,0] \is 1$
    \Statex
    \State \Comment{Recursion...}
    \Statex
    \For{$t \is 2$ to $T$}
    \State $R[0,t-1] \is 0$
    \For{$r_{t-1} \is 0$ to $(t-1)-1$}
    \State $R[0,t-1] \is R[0,t-1] +
    H(r_{t-1}+1)\,P[r_{t-1},t-2]\,R[r_{t-1},t-2]$
    \EndFor
    \State $Z \is R[0,t-1]$
    \For{$r_t \is 1$ to $t-1$}
    \State $R[r_t, t-1] \is (1 - H(r_t))\,P[r_t-1,t-2]\,R[r_t-1, t-2]$
    \State $Z \is Z + R[r_t,t-1]$
    \EndFor
    \Statex
    \State \Comment{$R[r_t,t-1]$ currently contains $P(r_t,
      x_{t-1} \g \bx_{1:t-2})$ and $Z$ contains $P(x_{t-1} \g \bx_{1:t-2})$}
    \Statex
    \For{$r_t \is 0$ to $t-1$}
    \State $R[r_t, t-1] \is R[r_t, t-1] \,/\, Z$
    \EndFor
    \EndFor
  \end{algorithmic}
\end{algorithm}

\begin{algorithm}[t]
  \caption[]{}
  \label{alg:conditional2}
  \begin{algorithmic}[0]
    \State \Comment{Base case (i.e., initialization)...}
    \Statex
    \State $R \is [ 1 ]$
    \Statex
    \State \Comment{Recursion...}
    \Statex
    \For{$t \is 2$ to $T$}
    \State $S \is \textrm{zeros}(t)$
    \For{$r \is 0$ to $(t-1)-1$}
    \State $S[r+1] \is (1 - H(r+1))\,P[r]\,R[r]$
    \EndFor
    \For{$r \is 0$ to $(t-1)-1$}
    \State $S[0] \is S[0] +
    H(r+1)\,P[r]\,R[r]$
    \EndFor
    \State $R \is S$
    \State $Z \is \textrm{sum}(S)$
    \Statex
    \State \Comment{$R[r_t]$ currently contains $P(r_t,
      x_{t-1} \g \bx_{1:t-2})$ and $Z$ contains $P(x_{t-1} \g \bx_{1:t-2})$}
    \Statex
    \For{$r \is 0$ to $t-1$}
    \State $R[r] \is R[r] \,/\, Z$
    \EndFor
    \EndFor
  \end{algorithmic}
\end{algorithm}

\subsubsection{Calculating $P(x_t \g \bx_{t-r_t:t-1})$}

Algorithm~\ref{alg:conditional} requires $P[r_{t-1}, t-2] \equiv
P(x_{t-1} \g \bx_{t-1-r_{t-1}:t-2})$ for all $1 \leq t \leq T$ and $0
\leq r_t \leq t-1$.

\comment{
If $x_t \sim \bphi_{\rho}$ for all $1 \leq t \leq T$ and $\bphi_{\rho}
\sim \Dir(\bphi_{\rho}; \beta \bn)$ for all $\rho$, as in
algorithm~\ref{alg:generative_process}, then
\begin{equation}
  P(x_t \g
  \bx_{t-r_t:t-1})
  = \frac{N_{x_t}^{t-r_t:t-1} + \beta
    n_{x_t}}{r_t + \beta},
\end{equation}
where $N_{x_t}^{t-r_t:t-1}$ is the number of observations in
$\bx_{t-r_t:t-1}$ with the same value as $x_t$. Letting
$\bN^{t-r_t:t-1}$ denote the $K$-dimensional vector whose
$k^{\textrm{th}}$ element is $N_k^{t-r_t:t-1}$, we can recursively
construct $\bN^{t-r_t:t-1} + \beta \bn$, i.e., the $K$-dimensional
vector of hyperparameters for the Dirichlet posterior over
$\bphi_{\rho}$ given the observations $\bx_{t-r_t:t-1}$ and the
$K$-dimensional vector of hyperparameters $\beta \bn$ for the prior
over $\bphi_{\rho}$, for all $1 \leq t \leq T$ and $0 \leq r_t \leq
t-1$ as shown in algorithm~\ref{alg:hyperparameters}. Once the set of
$T\,(T-1) \,/\, 2 + T$ hyperparameter vectors has been (recursively)
constructed, we can compute $P(x_t \g r_t, \bx_{1:t-1}) \teq P(x_t \g
\bx_{t-r_t:t-1}) = N_{x_t}^{t-r_t:t-1} + \beta n_{x_t} \,/\, r_t +
\beta$ as shown in algorithm~\ref{alg:predictive}.}

\begin{algorithm}[t]
  \caption[]{$\chi[r_t,t-1]$ is the $K$-dimensional vector of
    hyperparameters for the Dirichlet posterior over $\bphi_{\rho}$
    given observations $\bx_{t-r_t:t-1}$ and the $K$-dimensional
    vector of hyperparameters $\beta \bn$ for the prior over
    $\bphi_{\rho}$. For example, $\chi[2,3] = \beta \bn + \bzero^{x_2}
    + \bzero^{x_3}$, where $\bzero^{k}$ is a $K$-dimensional vector
    whose elements are all zero except for element $k$ which is one,
    because $t - 1 \teq 3 \implies t \teq 4$ and $r_4 \teq 2$.}
  \label{alg:hyperparameters}
  \begin{algorithmic}[0]
    \State \Comment{Base case (i.e., initialization)...}
    \Statex
    \State $\chi[0,0] \is \beta \bn$
    \State $\nu[0,0] \is \beta$
    \Statex
    \State \Comment{Recursion...}
    \Statex
    \For{$t \is 2$ to $T$}
    \State $\chi[0,t-1] \is \beta \bn$
    \State $\nu[0,t-1] \is \beta$
    \For{$r_t \is 1$ to $t-1$}
    \State $\chi[r_t,t-1] \is \chi[r_t-1,t-2] + \bzero^{x_{t-1}}$
    \State $\nu[r_t,t-1] \is \nu[r_t-1,t-2] + 1$
    \EndFor
    \EndFor
  \end{algorithmic}
\end{algorithm}

\begin{algorithm}[t]
  \caption[]{}
  \label{alg:hyperparameters2}
  \begin{algorithmic}[0]
    \State \Comment{Base case (i.e., initialization)...}
    \Statex
    \State $\bchi \is [\beta \bn ]$
    \State $\nu \is [\beta]$
    \Statex
    \State \Comment{Recursion...}
    \Statex
    \For{$t \is 2$ to $T$}
    \For{$r \is 0$ to $(t-1)-1$}
    \State $\bchi[r][x[t-2]] \is
    \bchi[r][x[t-2]] + 1$
    \State $\nu[r] \is \nu[r] + 1$
    \EndFor
    \State $\bchi.\textrm{prepend}(\beta \bn)$
    \State $\nu.\textrm{prepend}(\beta)$
    \EndFor
  \end{algorithmic}
\end{algorithm}

\begin{algorithm}[t]
  \caption[]{$P[r_t,t-1]$ is the predictive probability of observation
    $x_t$ given observations $\bx_{t-r_t:t-1}$. For example, $P[2,3] =
    P(x_4 \g \bx_{2:3})$ because $t -1 \teq 3 \implies t \teq 4$ and
    $r_4 \teq 2$.  $\chi[r_t,t-1]$ is the $K$-dimensional vector of
    hyperparameters for the Dirichlet posterior over $\bphi_{\rho}$
    given observations $\bx_{t-r_t:t-1}$ and the $K$-dimensional
    vector of hyperparameters $\beta \bn$ for the prior over
    $\bphi_{\rho}$.}
  \label{alg:predictive}
  \begin{algorithmic}[0]
    \For{$t \is 1$ to $T$}
    \For{$r_t \is 0$ to $t-1$}
    \State $P[r_t,t-1] \is \chi[r_t,t-1]_{x_t} \,/\, \nu[r_t,t-1]$
    \EndFor
    \EndFor
  \end{algorithmic}
\end{algorithm}

\begin{algorithm}[t]
  \caption[]{}
  \label{alg:predictive2}
  \begin{algorithmic}[0]
    \For{$t \is 1$ to $T$}
    \State $P \is \textrm{zeros}(t)$
    \For{$r \is 0$ to $t-1$}
    \State $P[r] \is \bchi[r][x[t-1]] \,/\, \nu[r]$
    \EndFor
    \EndFor
  \end{algorithmic}
\end{algorithm}

\begin{algorithm}[t]
  \caption[]{Combining algorithms~\ref{alg:hyperparameters}
    and~\ref{alg:predictive}.}
  \begin{algorithmic}[0]
    \State \Comment{Base case (i.e., initialization)...}
    \Statex
    \State $\chi[0,0] \is \beta \bn$
    \State $\nu[0,0] \is \beta$
    \State $P[0,0] \is n_{x_1}$
    \Statex
    \State \Comment{Recursion...}
    \Statex
    \For{$t \is 2$ to $T$}
    \State $\chi[0,t-1] \is \beta \bn$
    \State $\nu[0,t-1] \is \beta$
    \State $P[0,t-1] \is n_{x_t}$
    \For{$r_t \is 1$ to $t-1$}
    \State $\chi[r_t,t-1] \is \chi[r_t-1,t-2] + \bzero^{x_{t-1}}$
    \State $\nu[r_t,t-1] \is \nu[r_t-1,t-2] + 1$
    \State $P[r_t,t-1] \is \chi[r_t, t-1]_{x_t} \,/\, \nu[r_t,t-1]$
    \EndFor
    \EndFor
  \end{algorithmic}
\end{algorithm}

\begin{algorithm}[t]
\caption[]{}
  \begin{algorithmic}[0]
    \State \Comment{Base case (i.e., initialization)...}
    \Statex
    \State $\bchi \is [ \beta \bn ]$
    \State $\nu \is [\beta]$
    \State $P \is [n[x[0]]]$
    \Statex
    \State \Comment{Recursion...}
    \Statex
    \For{$t \is 2$ to $T$}
    \State $P \is \textrm{zeros}(t)$
    \For{$r \is 0$ to $(t-1)-1$}
    \State $\bchi[r][x[t-2]] \is
    \bchi[r][x[t-2]] + 1$
    \State $\nu[r] \is \nu[r] + 1$
    \State $P[r+1] \is \bchi[r][x[t-1]] \,/\, \nu[r]$
    \EndFor
    \State $\bchi.\textrm{prepend}(\beta \bn)$
    \State $\nu.\textrm{prepend}(\beta)$
    \State $P[0] \is \bchi[0][x[t-1]] \,/\, \nu[0]$
    \EndFor
  \end{algorithmic}
\end{algorithm}

\begin{algorithm}[t]
  \caption[]{Combining algorithms~\ref{alg:conditional},
    \ref{alg:hyperparameters}, and~\ref{alg:predictive}.}
  \label{alg:inference}
  \begin{algorithmic}[0]
    \State \Comment{Base case (i.e., initialization)...}
    \Statex
    \State $\chi[0,0] \is \beta \bn$
    \State $\nu[0,0] \is \beta$
    \State $P[0,0] \is n_{x_1}$
    \State $R[0,0] \is 1$
    \Statex
    \State \Comment{Recursion...}
    \Statex
    \For{$t \is 2$ to $T$}
    \State $\chi[0,t-1] \is \beta \bn$
    \State $\nu[0,t-1] \is \beta$
    \State $P[0,t-1] \is n_{x_t}$
    \State $R[0,t-1] \is 0$
    \For{$r_{t-1} \is 0$ to $(t-1)-1$}
    \State $R[0,t-1] \is R[0,t-1] +
    H(r_{t-1}+1)\,P[r_{t-1},t-2]\,R[r_{t-1},t-2]$
    \EndFor
    \State $Z \is R[0,t-1]$
    \For{$r_t \is 1$ to $t-1$}
    \State $\chi[r_t,t-1] \is \chi[r_t-1,t-2] + \bzero^{x_{t-1}}$
    \State $\nu[r_t,t-1] \is \nu[r_t-1,t-2] + 1$
    \State $P[r_t,t-1] \is \chi[r_t, t-1]_{x_t} \,/\, \nu[r_t,t-1]$
    \State $R[r_t, t-1] \is (1 - H(r_t))\,P[r_t-1,t-2]\,R[r_t-1, t-2]$
    \State $Z \is Z + R[r_t,t-1]$
    \EndFor
    \Statex
    \State \Comment{$R[r_t, t-1]$ currently contains $P(r_t, x_{t-1} \g
      \bx_{1:t-2})$ and $Z$ contains $P(x_{t-1} \g \bx_{1:t-2})$}
    \Statex
    \For{$r_t \is 0$ to $t-1$}
    \State $R[r_t, t-1] \is R[r_t, t-1] \,/\, Z$
    \EndFor
    \Statex
    \State \Comment{We \emph{actually} want $P(r_t \g \bx_{1:t}) \propto P(r_t, x_t \g
      \bx_{1:t-1}) = P(x_t \g r_t, \bx_{1:t-1})\,P(r_t \g
      \bx_{1:t-1})$}
    \Statex
    \EndFor
  \end{algorithmic}
\end{algorithm}

\begin{algorithm}[t]
  \caption[]{}
  \label{alg:inference2}
  \begin{algorithmic}[0]
    \State \Comment{Base case (i.e., initialization)...}
    \Statex
    \State $\bchi \is [\beta \bn]$
    \State $\nu \is [ \beta]$
    \State $P \is [n[x[0]]]$
    \State $R \is [ 1 ]$
    \Statex
    \State \Comment{Recursion...}
    \Statex
    \For{$t \is 2$ to $T$}
    \State $Q \is \textrm{zeros}(t)$
    \State $S \is \textrm{zeros}(t)$
    \For{$r \is 0$ to $(t-1)-1$}
    \State $\bchi[r][x[t-2]] \is \bchi[r][x[t-2]] + 1$
    \State $\nu[r] \is \nu[r] + 1$
    \State $Q[r+1] \is \bchi[r][x[t-1]] \,/\, \nu[r]$
    \State $S[r+1] \is (1 - H(r+1))\,P[r]\,R[r]$
    \EndFor
    \State $\bchi.\textrm{prepend}(\beta \bn)$
    \State $\nu.\textrm{prepend}(\beta)$
    \State $Q[0] \is \bchi[0][x[t-1]] \,/\, \nu[0]$
    \For{$r \is 0$ to $(t-1)-1$}
    \State $S[0] \is S[0] +
    H(r+1)\,P[r]\,R[r]$
    \EndFor
    \State $P \is Q$
    \State $R \is S$
    \State $Z \is \textrm{sum}(R)$
    \Statex
    \State \Comment{$R[r_t]$ currently contains $P(r_t,
      x_{t-1} \g \bx_{1:t-2})$ and $Z$ contains $P(x_{t-1} \g \bx_{1:t-2})$}
    \Statex
    \For{$r \is 0$ to $t-1$}
    \State $R[r] \is R[r] \,/\, Z$
    \EndFor
    \Statex
    \State \Comment{We \emph{actually} want $P(r_t \g \bx_{1:t}) \propto P(r_t, x_t \g
      \bx_{1:t-1}) = P(x_t \g r_t, \bx_{1:t-1})\,P(r_t \g
      \bx_{1:t-1})$}
    \Statex
    \EndFor
  \end{algorithmic}
\end{algorithm}

\section{Notes About Hazard Functions}

The hazard function $H(\tau)$ is defined as
\begin{equation}
  H(\tau) = \frac{f(\tau)}{1 - F(\tau)},
\end{equation}
where
$f(\tau)$ is a PDF (and can therefore be greater than one) such that
\begin{equation}
  P(a \leq \tau \leq b) = \int_a^b \textrm{d}\tau \, f(\tau)
\end{equation}
and $F(\tau)$ is the corresponding CDF, defined as
\begin{equation}
  F(\tau) = \int_{-\infty}^{\tau} \textrm{d}x\, f(x).
\end{equation}
The hazard function can therefore take on values that are greater than
one.

To sample a value $\tau$ from $f(\tau)$, note that
\begin{align}
  0 \leq F(\tau) = \int_{- \infty}^{\tau} \textrm{d}x \, f(x) \leq 1.
\end{align}
Any $u \sim U(0,1)$ therefore corresponds to $F(\tau)$ for some
$\tau$. Therefore $\tau \teq F^{-1}(u)$. In our case,
\begin{equation}
  H(\tau) = \frac{f(\tau)}{1 - F(\tau)} = \lambda.
\end{equation}
This corresponds to $f(\tau) \teq \lambda\,\exp{(-\lambda \tau)}$ and
$F(\tau) \teq 1 - \exp{(-\lambda \tau)}$.

\bibliography{changepoint}

\end{document}
